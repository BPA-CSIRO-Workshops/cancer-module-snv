%----------------------------------------------------------------------------------------
%	MODULE INFORMATION
%----------------------------------------------------------------------------------------

% Define the top matter
\setModuleTitle{Single Nucleotide Variant Call and Annotation}
\setModuleAuthors{%
  Matt Field \mailto{matt.field@anu.edu.au}\\
  Dan Andrews \mailto{dan.andrews@anu.edu.au}\\
  Velimir Gayevskiy \mailto{v.gayevskiy@garvan.org.au} \\
  Mathieu Bourgney \mailto{mathieu.bourgey@mcgill.ca}% 
}

\setModuleContributions{%
  Gayle Phillip \mailto{Sonika.Tyagi@agrf.org.au} \\
  Sonika Tyagi \mailto{gkphilip@unimelb.edu.au} \\
  Velimir Gayevskiy, Garvan Institute \mailto{v.gayevskiy@garvan.org.au}
}

%----------------------------------------------------------------------------------------
%	MODULE TITLE PAGE
%----------------------------------------------------------------------------------------

\chapter{\moduleTitle}

%----------------------------------------------------------------------------------------

\newpage

%----------------------------------------------------------------------------------------
%	LEARNING OUTCOMES
%----------------------------------------------------------------------------------------

\section{Key Learning Outcomes}

After completing this practical the trainee should be able to:

\begin{itemize}
  \item Prepare raw BAM alignments for variant detection 
  \item Perform QC measures on BAM files
  \item Understand and perform simple variant detection on paired NGS data 
  \item Add annotation information to raw variant calls
  \item Visualise variant calls using IGV
\end{itemize}

%----------------------------------------------------------------------------------------
%	MODULE RESOURCES
%----------------------------------------------------------------------------------------

\section{Resources You'll be Using}

\subsection{Tools Used}

\begin{description}[style=multiline,labelindent=0cm,align=left,leftmargin=1cm]
  \item[SAMTools] \hfill\\
    \url{http://sourceforge.net/projects/samtools/}
  \item[IGV] \hfill\\
    \url{http://www.broadinstitute.org/igv/}
  \item[Genome Analysis Toolkit] \hfill\\
    \url{http://www.broadinstitute.org/gatk/}
  \item[Picard] \hfill\\
    \url{http://picard.sourceforge.net/}
  \item[MuTect] \hfill\\
    \url{http://www.broadinstitute.org/cancer/cga/mutect/}
  \item[Strelka] \hfill\\
    \url{https://sites.google.com/site/strelkasomaticvariantcaller/}
  \item[VarScan2] \hfill\\
    \url{https://github.com/dkoboldt/varscan/}
  \item[Variant Effect Predictor] \hfill\\
    \url{http://www.ensembl.org/info/docs/tools/vep}
  \item[GEMINI] \hfill\\
    \url{http://gemini.readthedocs.org}
\end{description}

%------------------------------------------------

\subsection{Sources of Data}

\url{http://sra.dnanexus.com/studies/ERP001071}\\
\url{http://www.ncbi.nlm.nih.gov/pubmed/22194472}

%----------------------------------------------------------------------------------------

\newpage

%----------------------------------------------------------------------------------------
%	INTRODUCTION
%----------------------------------------------------------------------------------------

\section{Introduction}

This talk is based on Introduction to DNA-Seq processing for cancer data by Mathieu Bourgey, Ph.D  

*https://bitbucket.org/mugqic/muqpic\_pipelines*

The goal of this hands-on session is to present the main steps that are commonly used to process and to analyze cancer sequencing data. We will focus only on whole genome data and provide command lines that allow detecting Single Nucleotide Variants (SNV). This workshop will show you how to launch individual steps of a complete DNA-Seq SNV pipeline using cancer data.

In the second part of the tutorial we will also be using IGV to visualise and manually inspect candidate variant calls.

%----------------------------------------------------------------------------------------
%	THE ENVIRONMENT
%----------------------------------------------------------------------------------------

\section{Prepare the Environment}

We will use a dataset derived from whole genome sequencing of a 33-yr-old lung adenocarcinoma patient, who is a never-smoker and has no familial cancer history.

The data consists of whole genome sequencing of liver metastatic lung cancer (frozen), primary lung cancer (FFPE) and blood tissue of a lung adenocarcinoma patient (AK55).

The BAM alignment files are contained in the subdirectory called \texttt{alignment} and are located in the following subdirectories:

\begin{description}[style=multiline,labelindent=1.5cm,align=left,leftmargin=2.5cm]
  \item[\texttt{normal/normal.sorted.bam} and \texttt{normal/normal.sorted.bam.bai}] \hfill\\
  \item[\texttt{tumor/tumor.sorted.bam} and \texttt{tumor/tumor.sorted.bam.bai}] \hfill\\ 
\end{description}

These files are based on subsetting the whole genomes derived from blood and liver metastases to the first 10Mb of chromosome 4. This will allow our analyses to run in a sufficient time during the workshop, but it's worth being aware that this is less < 0.5\% of the genome which highlights the length of time and resourced required to perform cancer genomics on full genomes!

The initial structure of your folders should look like this:

\begin{verbatim}
-- alignment/	          # bam files 
  -- normal/              # The blood sample directory containing bam files 
  -- tumour/               # The tumour sample directory containing bam files \\
-- ref/                   # Contains reference genome files	     
-- commands.sh            # cheat sheet 
\end{verbatim}



\begin{steps}
Now we need to set some environment variables to save typing lengthy file paths over and over. Copy and paste the following commands into your terminal.
\begin{lstlisting}
export APP_ROOT=/home/trainee/snv/Applications

export IGVTOOLS_PATH=$PATH:$APP_ROOT/igvtools/

export PICARD_JAR=$APP_ROOT/picard/picard.jar

export SNPEFF_HOME=$APP_ROOT/snpeff/

export GATK_JAR=$APP_ROOT/gatk/GenomeAnalysisTK.jar

export BVATOOLS_JAR=$APP_ROOT/bvatools/bvatools-1.6-full.jar

export TRIMMOMATIC_JAR=$APP_ROOT/trimmomatic/trimmomatic-0.33.jar

export STRELKA_HOME=$APP_ROOT/strelka/

export MUTECT_JAR=$APP_ROOT/mutect/muTect-1.1.4.jar

export VARSCAN_JAR=$APP_ROOT/varscan/VarScan.v2.4.1.jar 

export REF=/home/trainee/snv/ref

export SNV_BASE=/home/trainee/snv

export JAVA7=/usr/lib/jvm/java-7-openjdk-amd64/jre/bin/java

export IGV=$APP_ROOT/igv/igv.sh
\end{lstlisting}
\end{steps}

\begin{steps}
Open the Terminal and go to the base \texttt/home/trainee/snv working directory:
\begin{lstlisting}
cd $SNV_BASE
\end{lstlisting}
\end{steps}

\begin{warning}
  All commands entered into the terminal for this tutorial should be from within the
  \textbf{\texttt/home/trainee/snv} directory.
\end{warning}

\begin{steps}
Check that the \texttt{alignment} directory contains the above-mentioned files by typing:
\begin{lstlisting}
ls alignment
\end{lstlisting}
\end{steps}


%----------------------------------------------------------------------------------------
%	BAM MANIPULATION
%----------------------------------------------------------------------------------------


\section{BAM Files}

Let's spend some time to explore bam files.

\subsection{Step 1: Exploring BAM files}

\begin{steps}
\begin{lstlisting}
samtools view alignment/normal/normal.sorted.bam | head -n4
\end{lstlisting}
\end{steps}

Here you have examples of alignment results.
A full description of the flags can be found in the SAM specification
\url{http://samtools.sourceforge.net/SAM1.pdf}

Another useful bit of information in the SAM is the CIGAR string.
It's the 6th column in the file. 

This column explains how the alignment was achieved.
 
        M == base aligns \textbf{but doesn't have to be a match.} A SNP will have an M even if it disagrees with the reference.\\
        I == Insertion\\
        D == Deletion\\
        S == soft-clips. These are handy to find un removed adapters, viral insertions, etc.

An in depth explanation of the CIGAR can be found \url{http://genome.sph.umich.edu/wiki/SAM}

The exact details of the cigar string can be found in the SAM spec as well.


We won't go into too much detail at this point since we want to concentrate on cancer specific issues now.


Now, you can try using picards explain flag site to understand what is going on with your reads
\url{http://broadinstitute.github.io/picard/explain-flags.html}

\begin{questions} 
There are 3 unique flags, what do they mean? The flag is the second column.
\end{questions}
\begin{answer}
\textbf{129:}\\ 
    read paired\\
    second in pair\\ \\
\textbf{113:}\\
    read paired\\
    read reverse strand\\
    mate reverse strand\\
    first in pair\\ \\
\textbf{161:}\\
    read paired\\
    mate reverse strand\\
    second in pair\\
\end{answer}

There are lots of possible different flags, let's look at a few more
\begin{lstlisting}
samtools view alignment/normal/normal.sorted.bam | head -n100
\end{lstlisting}


\begin{questions} 
Let's take the last one, which looks properly paired and find it's mate pair. \\
HINT: Instead of using 'head' what unix command could we pipe the output to?
HINT2: Once we've found both reads to stop the command running we type CTRL-C
\end{questions}
\begin{answer}
\begin{lstlisting}
samtools view alignment/normal/normal.sorted.bam | grep HWI-ST478_0133:4:2205:14675:32513
\end{lstlisting}
\end{answer}

\begin{questions} 
Using the cigar string, what can we tell about the alignment of the mate pair?
\end{questions}
\begin{answer}
The mate pair has a less convincing alignment with two insertions and soft clipping reported.
\end{answer}

\begin{questions} 
How might the alignment information from the original read be used by the aligner?
\end{questions}
\begin{answer}
Even though the alignment of the mate pair is questionable the presence of it's properly paired mate helps the aligner in deciding where to put the less-certain read. 
\end{answer}

You can use samtools to filter reads as well.

\begin{questions} 
How many reads mapped and unmapped were there? \\
HINT: Look at the samtools view help menu by typing samtools view without any arguments
\end{questions}
\begin{answer}
\begin{lstlisting}
samtools view -c -f4 alignment/normal/normal.sorted.bam
\end{lstlisting}
77229
\begin{lstlisting}
samtools view -c -F4 alignment/normal/normal.sorted.bam
\end{lstlisting}
22972373
\end{answer}

\subsection{Step 2: Pre-processing: Indel Realignment}
The first step for this is to realign around indels and snp dense regions.\\
The Genome Analysis toolkit has a tool for this called IndelRealigner. \\
It basically runs in 2 steps: \\
   1. Find the targets \\
   2. Realign them \\

\begin{lstlisting}
$JAVA7 -Xmx2G  -jar ${GATK_JAR} \
  -T RealignerTargetCreator \
  -R ${REF}/human_g1k_v37.fasta \
  -o alignment/normal/realign.intervals \
  -I alignment/normal/normal.sorted.bam \
  -I alignment/tumour/tumour.sorted.bam \
  -L ${REF}/human_g1k_v37.intervals

$JAVA7 -Xmx2G -jar ${GATK_JAR} \
  -T IndelRealigner \
  -R ${REF}/human_g1k_v37.fasta \
  -targetIntervals alignment/normal/realign.intervals \
  --nWayOut .realigned.bam \
  -I alignment/normal/normal.sorted.bam \
  -I alignment/tumour/tumour.sorted.bam \
  -L ${REF}/human_g1k_v37.intervals

  mv normal.sorted.realigned.ba* alignment/normal/
  mv tumour.sorted.realigned.ba* alignment/tumour/
\end{lstlisting}


\begin{note}
Explanation of parameters
\begin{description}[style=multiline,labelindent=0cm,align=right,leftmargin=\descriptionlabelspace,rightmargin=1.5cm,font=\ttfamily]
 \item[-I] BAM file(s)
 \item[-T] GATK algorithm to run
 \item[-R] the reference genome used for mapping (b37 from GATK here)
 \item[-jar] Path to GATK jar file
 \item[-L] Genomic intervals to operate on
\end{description}
\end{note}

\begin{questions} 
Why did we use both normal and tumor together?
\end{questions}
\begin{answer}
Because if a region needs realignment, maybe one of the sample in the pair has less reads or was excluded from the target creation. \\
This makes sure the normal and tumor are all in-sync for the somatic calling step. 
\end{answer}

\begin{questions} 
How many regions did it think needed cleaning ? 
\end{questions}
\begin{answer}
\begin{lstlisting}
wc -l alignment/normal/realign.intervals -> 27300
\end{lstlisting}
\end{answer}

Indel Realigner also makes sure the called deletions are left aligned when there is a microsatellite or homopolymer.

\begin{verbatim}
This
ATCGAAAA-TCG
into
ATCG-AAAATCG

or
ATCGATATATATA--TCG
into
ATCG--ATATATATATCG
\end{verbatim}


\begin{questions} 
Why it is important ? 
\end{questions}
\begin{answer}
This makes it easier for down stream analysis tools

For NGS analysis, the convention is to left align indels. 

This is only really needed when calling variants with legacy locus-based tools such as samtools or GATK UnifiedGenotyper. Otherwise you will have worse performance and accuracy.

With more sophisticated tools (like GATK HaplotypeCaller) that involve reconstructing haplotypes (eg through reassembly), the problem of multiple valid representations is handled internally and does not need to be corrected explicitly.
\end{answer}

\subsection{Step 3: Pre-processing: Fixmates}

Some read entries don't have their mate information written properly. \\
We use Picard to do this: 

\begin{lstlisting}
$JAVA7 -Xmx2G -jar ${PICARD_JAR} FixMateInformation \
  VALIDATION_STRINGENCY=SILENT \
  CREATE_INDEX=true \
  SORT_ORDER=coordinate \
  MAX_RECORDS_IN_RAM=500000 \
  INPUT=alignment/normal/normal.sorted.realigned.bam \
  OUTPUT=alignment/normal/normal.matefixed.bam

$JAVA7 -Xmx2G -jar ${PICARD_JAR} FixMateInformation \
  VALIDATION_STRINGENCY=SILENT \
  CREATE_INDEX=true \
  SORT_ORDER=coordinate \
  MAX_RECORDS_IN_RAM=500000 \
  INPUT=alignment/tumour/tumour.sorted.realigned.bam \
  OUTPUT=alignment/tumour/tumour.matefixed.bam

\end{lstlisting}

\subsection{Step 4: Pre-processing: Mark Duplicates}

\begin{questions}
What are duplicate reads ?
\end{questions}
\begin{answer}
Different read pairs representing the same initial DNA fragment.
\end{answer}

\begin{questions} 
What are they caused by ?
\end{questions}
\begin{answer}
PCR reactions (PCR duplicates) \\
Some clusters that are thought of being separate in the flowcell but are the same (optical duplicates)
\end{answer}

\begin{questions} 
What are the ways to detect them ?
\end{questions}
\begin{answer}
Picard and samtools uses the alignment positions: \\
   - Both 5' ends of both reads need to have the same positions. \\ 
   - Each reads have to be on the same strand as well. \\ \\
Another method is to use a kmer approach: \\
   - take a part of both ends of the fragment \\
   - build a hash table  \\
   - count the similar hits \\ \\
Brute force, compare all the sequences.
\end{answer}

Here we will use picards approach:
\begin{lstlisting}
$JAVA7 -Xmx2G -jar ${PICARD_JAR} MarkDuplicates \
  REMOVE_DUPLICATES=false \
  CREATE_MD5_FILE=true \
  VALIDATION_STRINGENCY=SILENT \
  CREATE_INDEX=true \
  INPUT=alignment/normal/normal.matefixed.bam \
  OUTPUT=alignment/normal/normal.sorted.dup.bam \
  METRICS_FILE=alignment/normal/normal.sorted.dup.metrics

$JAVA7 -Xmx2G -jar ${PICARD_JAR} MarkDuplicates \
  REMOVE_DUPLICATES=false \
  CREATE_MD5_FILE=true \
  VALIDATION_STRINGENCY=SILENT \
  CREATE_INDEX=true \
  INPUT=alignment/tumour/tumour.matefixed.bam \
  OUTPUT=alignment/tumour/tumour.sorted.dup.bam \
  METRICS_FILE=alignment/tumour/tumour.sorted.dup.metrics
\end{lstlisting}


We can look in the metrics output to see what happened.

\begin{lstlisting}
less alignment/normal/normal.sorted.dup.metrics
\end{lstlisting}

\begin{questions} 
What percent of reads are duplicates?
\end{questions}
\begin{answer}
0.046996\%
\end{answer}

\begin{questions} 
Often, we have multiple libraries and when this occurs separate measures are calculated for each library. Why is this important to do not combine everything ?
\end{questions}
\begin{answer}
Each library represents a set of different DNA fragments.

Each library involves different PCR reactions

So PCR duplicates can not occur between fragment of two different libraries.

But similar fragment could be found between libraries when the coverage is high.
\end{answer}


\subsection{Step 4: Pre-processing: Base Quality Recalibration}

\begin{questions}
Why do we need to recalibrate base quality scores ?
\end{questions}
\begin{answer}
The vendors tend to inflate the values of the bases in the reads.
The recalibration tries to lower the scores of some biased motifs for some technologies.
\end{answer}

It runs in 2 steps, \\
1- Build covariates based on context and known snp sites \\
2- Correct the reads based on these metrics

GATK BaseRecalibrator:

\begin{lstlisting}
for i in normal tumour
do
  $JAVA7 -Xmx2G -jar ${GATK_JAR} \
    -T BaseRecalibrator \
    -nct 2 \
    -R ${REF}/human_g1k_v37.fasta \
    -knownSites ${REF}/dbSnp-138_chr4.vcf \
    -L 4:1-10000000 \
    -o alignment/${i}/${i}.sorted.dup.recalibration_report.grp \
    -I alignment/${i}/${i}.sorted.dup.bam

    $JAVA7 -Xmx2G -jar ${GATK_JAR} \
      -T PrintReads \
      -nct 2 \
      -R ${REF}/human_g1k_v37.fasta \
      -BQSR alignment/${i}/${i}.sorted.dup.recalibration_report.grp \
      -o alignment/${i}/${i}.sorted.dup.recal.bam \
      -I alignment/${i}/${i}.sorted.dup.bam
done
\end{lstlisting}

\section{BAM QC}

Once your whole bam is generated, it's always a good thing to check the data again to see if everything makes sense.

\subsection{Step 1: BAM QC: Compute Coverage}
If you have data from a capture kit, you should see how well your targets worked. Both GATK and BVATools have depth of coverage tools. \\
Here we'll use the GATK one
\begin{lstlisting}

for i in normal tumour
do
  $JAVA7  -Xmx2G -jar ${GATK_JAR} \
    -T DepthOfCoverage \
    --omitDepthOutputAtEachBase \
    --summaryCoverageThreshold 10 \
    --summaryCoverageThreshold 25 \
    --summaryCoverageThreshold 50 \
    --summaryCoverageThreshold 100 \
    --start 1 --stop 500 --nBins 499 -dt NONE \
    -R ${REF}/human_g1k_v37.fasta \
    -o alignment/${i}/${i}.sorted.dup.recal.coverage \
    -I alignment/${i}/${i}.sorted.dup.recal.bam \
    -L 4:1-10000000
done
\end{lstlisting}

\begin{note}
Explanation of parameters
\begin{description}[style=multiline,labelindent=0cm,align=right,leftmargin=\descriptionlabelspace,rightmargin=1.5cm,font=\ttfamily]
 \item[omitBaseOutput] Do not output depth of coverage at each base
 \item[summaryCoverageThreshol] Coverage threshold (in percent) for summarizing statistics
 \item[dt] down sampling
 \item[L] Genomic intervals to operate on
\end{description}
\end{note}

Coverage is expected to be ~25x in these project
Look at the coverage:

\begin{lstlisting}
less -S alignment/normal/normal.sorted.dup.recal.coverage.sample_interval_summary
less -S alignment/tumour/tumour.sorted.dup.recal.coverage.sample_interval_summary
\end{lstlisting}

\begin{questions}
Is the coverage fit with the expectation ?
\end{questions}
\begin{answer}
Yes the mean coverage of the region is 25x:

summaryCoverageThreshold is a useful function to see if your coverage is uniform.
 
Another way is to compare the mean to the median. If both are quite different that means something is wrong in your coverage.
\end{answer}

\subsection{Step 2: BAM QC: Insert Size}
It corresponds to the size of DNA fragments sequenced.

Different from the gap size (= distance between reads) !

These metrics are computed using Picard:

\begin{lstlisting}
for i in normal tumour
do
  $JAVA7 -Xmx2G -jar ${PICARD_JAR} CollectInsertSizeMetrics \
    VALIDATION_STRINGENCY=SILENT \
    REFERENCE_SEQUENCE=${REF}/human_g1k_v37.fasta \
    INPUT=alignment/${i}/${i}.sorted.dup.recal.bam \
    OUTPUT=alignment/${i}/${i}.sorted.dup.recal.metric.insertSize.tsv \
    HISTOGRAM_FILE=alignment/${i}/${i}.sorted.dup.recal.metric.insertSize.histo.pdf \
    METRIC_ACCUMULATION_LEVEL=LIBRARY
done
\end{lstlisting}

look at the output

\begin{lstlisting}
less -S alignment/normal/normal.sorted.dup.recal.metric.insertSize.tsv
less -S alignment/tumour/tumour.sorted.dup.recal.metric.insertSize.tsv
\end{lstlisting}

\begin{questions}
How do the two libraries compares? 
\end{questions}
\begin{answer}
The tumour sample has a larger median insert size than the normal sample (405 vs 329) 
\end{answer}

\subsection{Step 3: BAM QC: Alignment metrics}
It tells you if your sample and you reference fit together

For the alignment metrics, samtools flagstat is very fast but with bwa-mem since some reads get broken into pieces, the numbers are a bit confusing. 

We prefer the Picard way of computing metrics:
\begin{lstlisting}
for i in normal tumour
do
  $JAVA7 -Xmx2G -jar ${PICARD_JAR} CollectAlignmentSummaryMetrics \
    VALIDATION_STRINGENCY=SILENT \
    REFERENCE_SEQUENCE=${REF}/human_g1k_v37.fasta \
    INPUT=alignment/${i}/${i}.sorted.dup.recal.bam \
    OUTPUT=alignment/${i}/${i}.sorted.dup.recal.metric.alignment.tsv \
    METRIC_ACCUMULATION_LEVEL=LIBRARY
done
\end{lstlisting}

explore the results

\begin{lstlisting}
less -S alignment/normal/normal.sorted.dup.recal.metric.alignment.tsv
less -S alignment/tumour/tumour.sorted.dup.recal.metric.alignment.tsv
\end{lstlisting}

\begin{questions}
Do you think the sample and the reference genome fit together ?
\end{questions}
\begin{answer}
Yes, 99\% of the reads have been aligned \\
Usually, we consider:  \\
   - A good alignment if > 85\% \\
   - Reference assembly issues if [60-85]\% \\
   - Probably a mismatch between sample and ref if < 60 \%
\end{answer}




\newpage

%----------------------------------------------------------------------------------------
%	VARIANT CALLING
%----------------------------------------------------------------------------------------

\section{Variant Calling}

Most of SNV caller use either a Baysian, a threshold or a t-test approach to do the calling

 Here we will try 3 variant callers.\\
- Varscan 2 \\
- MuTecT \\
- Strelka

Other candidates \\
- Virmid \\
- Somatic sniper

many, MANY others can be found here:
\url{https://www.biostars.org/p/19104/}


In our case, let's create a new work directory to start with (from base directory):

\begin{lstlisting}
cd $SNV_BASE
mkdir variant_calling
\end{lstlisting}

varscan 2
VarScan calls somatic variants (SNPs and indels) using a heuristic method and a statistical test based on the number of aligned reads supporting each allele.

Varscan somatic caller expects both a normal and a tumour file in SAMtools pileup format from sequence alignments in binary alignment/map (BAM) format. To build a pileup file, you will need:

- A SAM/BAM file ("myData.bam") that has been sorted using the sort command of SAMtools.
- The reference sequence ("reference.fasta") to which reads were aligned, in FASTA format.
- The SAMtools software package.


\begin{lstlisting}
for i in normal tumour
do
samtools mpileup -L 1000 -B -q 1 \
  -f ${REF}/human_g1k_v37.fasta \
  -r 4:1-10000000 \
  alignment/${i}/${i}.sorted.dup.recal.bam \
  > variant_calling/${i}.mpileup
done

$JAVA7 -Xmx2G -jar ${VARSCAN_JAR} \
somatic variant_calling/normal.mpileup \
variant_calling/tumour.mpileup \
variant_calling/varscan \
--output-vcf 1 \
--strand-filter 1 \
--somatic-p-value 0.001 
\end{lstlisting}

Notes on samtools arguments
\begin{description}[style=multiline,labelindent=0cm,align=right,leftmargin=\descriptionlabelspace,rightmargin=1.5cm,font=\ttfamily]
	\item[-L] = max per-sample depth for INDEL calling [1000] ; 
	\item[-B] = disable BAQ (per-Base Alignment Quality) ; 
	\item[-q] = skip alignments with mapQ smaller than 1 ; 
	\item[-g] = generate genotype likelihoods in BCF format
\end{description} 

Notes on bcftools arguments
\begin{description}[style=multiline,labelindent=0cm,align=right,leftmargin=\descriptionlabelspace,rightmargin=1.5cm,font=\ttfamily]
	\item[-v] = output potential variant sites only
	\item[-c] = SNP calling (force –e : likelihood based analyses)
	\item[-g] = call genotypes at variant sites
\end{description}

Now let's try a different variant caller, MuTect \\

\begin{note} 
Note MuTecT only works with Java 6, 7 will give you an error \\
if you get "Comparison method violates its general contract! \\
you used java 7"
\end{note}

\begin{lstlisting}
java -Xmx2G -jar ${MUTECT_JAR} \
  -T MuTect \
  -R ${REF}/human_g1k_v37.fasta \
  -dt NONE -baq OFF --validation_strictness LENIENT -nt 2 \
  --dbsnp ${REF}/dbSnp-138_chr4.vcf \
  --input_file:normal alignment/normal/normal.sorted.dup.recal.bam \
  --input_file:tumor alignment/tumour/tumour.sorted.dup.recal.bam \
  --out variant_calling/mutect.call_stats.txt \
  --coverage_file variant_calling/mutect.wig.txt \
  -pow variant_calling/mutect.power \
  -vcf variant_calling/mutect.vcf \
  -L 4:1-10000000
\end{lstlisting}

And finally let's try Illumina's Strelka

\begin{lstlisting}
cp ${STRELKA_HOME}/etc/strelka_config_bwa_default.ini .

sed 's/isSkipDepthFilters =.*/isSkipDepthFilters = 1/g' -i strelka_config_bwa_default.ini

${STRELKA_HOME}/bin/configureStrelkaWorkflow.pl \
  --normal=alignment/normal/normal.sorted.dup.recal.bam \
  --tumor=alignment/tumour/tumour.sorted.dup.recal.bam \
  --ref=${REF}/human_g1k_v37.fasta \
  --config=${SNV_BASE}/strelka_config_bwa_default.ini \
  --output-dir=variant_calling/strelka/

  cd variant_calling/strelka/
  make -j2
  cd ../..

  cp variant_calling/strelka/results/passed.somatic.snvs.vcf variant_calling/strelka.vcf
\end{lstlisting}

Now we have variants from all three methods. Let's compress and index the vcfs for future visualisation.

\begin{lstlisting}
for i in variant_calling/*.vcf;do bgzip -c $i > $i.gz ; tabix -p vcf $i.gz;done
\end{lstlisting}

Let's look at a compressed vcf.

\begin{lstlisting}
zless -S variant_calling/varscan.snp.vcf.gz
\end{lstlisting}

Details on the spec can be found here:
\url{http://vcftools.sourceforge.net/specs.html}

Fields vary from caller to caller.
 
Some values are are almost always there: \\
   - The ref vs alt alleles, \\
   - variant quality (QUAL column) \\
   - The per-sample genotype (GT) values.

Note on vcf fields
\begin{description}[style=multiline,labelindent=0cm,align=right,leftmargin=\descriptionlabelspace,rightmargin=1.5cm,font=\ttfamily]
	\item[DP] = "Raw read depth"
	\item[GT] = "Genotype" 
	\item[PL] = "List of Phred-scaled genotype likelihoods" (min is better) 
	\item[DP] = "\# high-quality bases" 
	\item[SP] = "Phred-scaled strand bias P-value"  
	\item[GQ] = "Genotype Quality"
\end{description}

\begin{questions}
Looking at the three vcf files, how can we detect only somatic variants?
\end{questions}
\begin{answer}
some commands to find somatic variant in the vcf file

varscan
\begin{lstlisting}
grep SOMATIC variant_calling/varscan.snp.vcf 
\end{lstlisting}

MuTecT
\begin{lstlisting}
grep -v REJECT variant_calling/mutect.vcf | grep -v "^#"
\end{lstlisting}

Strelka
\begin{lstlisting}
grep -v "^#" variant_calling/strelka.vcf
\end{lstlisting}
\end{answer}


\newpage

\section{Variant Visualisation}

The Integrative Genomics Viewer (IGV) is an efficient visualization tool for interactive exploration of large genome datasets. 

Before jumping into IGV, we'll generate a track IGV can use to plot coverage:


\begin{lstlisting}
for i in normal tumour
do
  $JAVA7 -jar ${IGVTOOLS_PATH}/igvtools.jar count \
    -f min,max,mean \
    alignment/${i}/${i}.sorted.dup.recal.bam \
    alignment/${i}/${i}.sorted.dup.recal.bam.tdf \
    b37
done
\end{lstlisting}

Open IGV
\begin{lstlisting}
$IGV
\end{lstlisting}

Then:
   1. Chose the reference genome corresponding to those use for alignment (b37) \\
   2. Load bam files (tumour.sorted.dup.recal.bam and normal.sorted.dup.recal.bam) \\
   3. Load vcf files (from variant\_Calling directory)

Explore/play with the data: \\ 
   -find germline variants \\
   -find somatic variants \\
   -Look around...

\newpage

%----------------------------------------------------------------------------------------
%	VARIANT ANNOTATION
%----------------------------------------------------------------------------------------

\section{Variant Annotation}

Following variant calling, we end up with a VCF file of genomic coordinates with the genotype(s) and quality information for each variant. By itself, this information is not much use to us unless there is a specific genomic location we are interested in. Generally, we next want to annotate these variants to determine whether they impact any genes and if so what is their level of impact (e.g. are they causing a premature stop codon gain or are they likely less harmful missense mutations).

The sections above have dealt with calling somatic variants from the first 10Mb of chromosome 4. This is important in finding variants that are unique to the tumour sample(s) and may have driven both tumour growth and/or metastasis. An important secondary question is whether the germline genome of the patient contains any variants that may have contributed to the development of the initial tumour through predisposing the patient to cancer. These variants \textit{may not} be captured by somatic variant analysis as their allele frequency may not change in the tumour genome compared with the normal.

For this section, we will use \textbf{all} variants from the first 60Mb of chromosome 5 that have been pre-generated using the GATK HaplotypeCaller variant caller on both the normal and tumour genomes. The output of this was GVCF files which were fed into GATK GenotypeGVCFs to produce a merged VCF file. We will use this pre-generated file as we are primarily interested in the annotation of variants rather than their generation. The annotation method we will use is called \textbf{Variant Effect Predictor} or VEP for short and is available from Ensembl here: \url{http://ensembl.org/info/docs/tools/vep/index.html}.

\begin{steps}
Our pre-generated VCF file is located in the \texttt{variants} folder. Let's have a quick look at the variants:
\begin{lstlisting}
zless variants/HC.chr5.60Mb.vcf.gz
\end{lstlisting}
\end{steps}

Notice how there are two genotype blocks at the end of each line for the normal (Blood) and tumour (liverMets) samples.

Let's now run VEP on this VCF file to annotate each variant with its impact(s) on the genome.

\begin{steps}
\begin{lstlisting}[breaklines=true,breakatwhitespace=false]
perl Applications/ensembl-tools/scripts/variant_effect_predictor/variant_effect_predictor.pl -i variants/HC.chr5.60Mb.vcf.gz --vcf -o variants/HC.chr5.60Mb.vep.vcf --stats_file variants/HC.chr5.60Mb.vep.html --format vcf --offline -fork 4 --fasta ref/human_g1k_v37.fasta --fields Consequence,Codons,Amino_acids,Gene,SYMBOL,Feature,EXON,PolyPhen,SIFT,Protein_position,BIOTYPE
\end{lstlisting}
\end{steps}

VEP will take approximately 10 minutes to run and once it is finished you will have a new VCF file with all of the information in the input file but with added annotations in the INFO block. VEP also produces an HTML report summarising the distribution and impact of variants identified.

\begin{steps}
Once VEP is done running, let's first look at the HTML report it produced with the following command:
\begin{lstlisting}
firefox variants/HC.chr5.60Mb.vep.html
\end{lstlisting}
\end{steps}

This report shows information on the VEP run, the number of variants, the classes of variants detected, the variant consequences and the distributions of variants through the genome. Close Firefox to resume the terminal prompt.

\begin{steps}
Now let's look at the variant annotations that VEP has added to the VCF file by focussing on a single variant. Let's fetch the same variant from the original VCF file and the annotated VCF file to see what has been changed.
\begin{lstlisting}
zcat variants/HC.chr5.60Mb.vcf.gz | grep '5\s174106\s'
grep '5\s174106\s' variants/HC.chr5.60Mb.vep.vcf
\end{lstlisting}
\end{steps}

These commands give us the original variant:

\begin{lstlisting}[breaklines=true,breakatwhitespace=false]
5	174106	.	G	A	225.44	.	AC=2;AF=0.500;AN=4;BaseQRankSum=1.22;ClippingRankSum=0.811;DP=21;FS=0.000;GQ_MEAN=127.00;GQ_STDDEV=62.23;MLEAC=2;MLEAF=0.500;MQ=60.00;MQ0=0;MQRankSum=0.322;NCC=0;QD=10.74;ReadPosRankSum=0.377;SOR=0.446	GT:AD:DP:GQ:PL	0/1:7,6:13:99:171,0,208	0/1:5,3:8:83:83,0,145
\end{lstlisting}

and the same variant annotated is:

\begin{lstlisting}[breaklines=true,breakatwhitespace=false]
5	174106	.	G	A	225.44	.	AC=2;AF=0.500;AN=4;BaseQRankSum=1.22;ClippingRankSum=0.811;DP=21;FS=0.000;GQ_MEAN=127.00;GQ_STDDEV=62.23;MLEAC=2;MLEAF=0.500;MQ=60.00;MQ0=0;MQRankSum=0.322;NCC=0;QD=10.74;ReadPosRankSum=0.377;SOR=0.446;CSQ=missense_variant|cGg/cAg|R/Q|ENSG00000153404|PLEKHG4B|ENST00000283426|16/18|||1076|protein_coding,non_coding_transcript_exon_variant&non_coding_transcript_variant|||ENSG00000153404|PLEKHG4B|ENST00000504041|5/8||||retained_intron	GT:AD:DP:GQ:PL	0/1:7,6:13:99:171,0,208	0/1:5,3:8:83:83,0,145
\end{lstlisting}

You can see that VEP has added:

\begin{lstlisting}[breaklines=true,breakatwhitespace=false]
CSQ=missense_variant|cGg/cAg|R/Q|ENSG00000153404|PLEKHG4B|ENST00000283426|16/18|||1076|protein_coding,non_coding_transcript_exon_variant&non_coding_transcript_variant|||ENSG00000153404|PLEKHG4B|ENST00000504041|5/8||||retained_intron
\end{lstlisting}

This is further composed of two annotations for this variant:

\begin{lstlisting}[breaklines=true,breakatwhitespace=false]
missense_variant|cGg/cAg|R/Q|ENSG00000153404|PLEKHG4B|ENST00000283426|16/18|||1076|protein_coding
\end{lstlisting}

and

\begin{lstlisting}[breaklines=true,breakatwhitespace=false]
non_coding_transcript_exon_variant&non_coding_transcript_variant|||ENSG00000153404|PLEKHG4B|ENST00000504041|5/8||||retained_intron
\end{lstlisting}

The first of these is saying that this variant is a missense variant in the gene PLEKHG4B for the transcript ENST00000283426 and the second that it is also a non\_coding\_transcript\_exon\_variant in the transcript ENST00000504041.

%----------------------------------------------------------------------------------------
%	VARIANT FILTRATION
%----------------------------------------------------------------------------------------

\section{Variant Filtration}

We now have a VCF file where each variant has been annotated with one or more impacts for one or more genes. In a typical whole cancer genome, you will have about 4-5 million variants, and therefore rows, in a VCF file which takes up gigabytes of space. In our small example, we have just 100,000 variants which is already too large to make any kind of meaningful sense out of by just opening up the VCF file in a text editor. We need a solution that allows us to perform intelligent queries on our variants to search this mass of noise for the signal we are interested in.

Luckily, such a free tool exists and is called GEMINI. GEMINI takes as an input your annotated VCF file and creates a database file which it can then query using Structured Query Language (SQL) commands. Not only does GEMINI make your variants easily searchable, it also brings in many external annotations to add more information about your variants (such as their frequencies in international databases).

\begin{steps}
To get started with GEMINI, let's make a database out of our annotated VCF file.
\begin{lstlisting}
gemini load -v variants/HC.chr5.60Mb.vep.vcf --cores 4 --skip-gerp-bp --skip-cadd -t VEP variants/HC.chr5.60Mb.vep.vcf.db
\end{lstlisting}
\end{steps}

This will take approximately 10 minutes. You will see a few errors due to multiallelic sites, normally these sites are decomposed and normalized before creating the GEMINI database but this is outside the scope of this workshop.

Once the database has been created let's run a basic query to see what kind of information we get out of GEMINI.

\begin{steps}
\begin{lstlisting}
gemini query -q "SELECT *, (gts).(*), (gt_types).(*), (gt_depths).(*), (gt_ref_depths).(*), (gt_alt_depths).(*), (gt_quals).(*) FROM variants LIMIT 10;" --header variants/HC.chr5.60Mb.vep.vcf.db
\end{lstlisting}
\end{steps}

This will output a bunch of ordered information for your query to the command line, this is usually saved to a TSV file and opened in a spreadsheet as we will do for the next query. In the mean time, let's dissect this query to understand the syntax we need to use to filter our variants. First, we have a SELECT statement which simply specifies that we want to select data from the database. The following comma-separated values are the columns that we want to output from the database, in this case we are selecting all columns with the star character and then all sub-columns for each sample with the other values. Then, we have a "FROM variants" statement which is specifying the table within the database that we want to fetch information from. Finally, the "LIMIT 10" statement specifies that no more than 10 rows should be returned. In summary then, we are asking for all columns for 10 rows from the table \texttt{variants}. If you haven't used SQL before don't worry, the GEMINI website is very helpful and provides many examples for how to query your database.

Let's now perform a more interesting query to find variants that have a medium or high impact on a gene and are rare or not present in existing international allele frequency databases. We will save the output of this query to a file and open it up in a spreadsheet.

\begin{steps}
\begin{lstlisting}
gemini query -q "SELECT *, (gts).(*), (gt_types).(*), (gt_depths).(*), (gt_ref_depths).(*), (gt_alt_depths).(*), (gt_quals).(*) FROM variants WHERE (impact_severity = 'HIGH' OR impact_severity = 'MED') AND (aaf_1kg_all < 0.01 OR aaf_1kg_all is null) AND (aaf_esp_all < 0.01 OR aaf_esp_all is null) AND (aaf_exac_all < 0.01 OR aaf_exac_all is null);" --header variants/HC.chr5.60Mb.vep.vcf.db > variants/gemini-result.tsv
\end{lstlisting}
\end{steps}

Notice that we have added a WHERE statement which restricts the rows that are returned based on values that we specify for specific columns. Here, we are asking to return variants where their impact on the gene (impact\_severity column) is medium or high and the allele frequency in 1000Genomes, ESP and EXaC is less than 1\% or the variant is not present in any of these databases.

\begin{steps}
Now let's open the result in a spreadsheet to look at the annotations:
\begin{lstlisting}
libreoffice --calc variants/gemini-result.tsv
Tick the "Tab" under "Separated by" on the dialog window that comes up.
\end{lstlisting}
\end{steps}

You can see that the first 14 columns contain information on the variant including its location, ref, alt, dbSNP ID, quality and type. Slowly scroll to the right and look at the columns of data that are provided. Most importantly, column BD includes the gene this variant impacts, BN the impact itself and BP the impact severity. Scroll towards the end of the spreadsheet until you get to columns ED and EE, these contain the genotype for each of the samples. Columns EH and EI contain the total depth for each variant in each sample and the 4 following columns contain the reference and alternate depths for each sample. Finally, columns EN and EO contain the genotype qualities (from the GQ field in the VCF) for each sample. As you scroll back and forth through this spreadsheet, you will see that GEMINI brings in information from a variety of sources including: OMIM, ClinVar, GERP, PolyPhen 2, SIFT, ESP, 1000 Genomes, ExAC, ENCODE, CADD and more! We are only looking at a small number of variants from the start of a chromosome so not many of these annotations will be present but in a full genome database they are incredibly useful.

GEMINI allows you to filter your variants based on any column that you see in this results file. For example, you may want all variants in a specific gene, in which case you would simply add "WHERE gene = 'BRCA1'" to your query. For complete documentation with many examples of queries, see the GEMINI documentation here: \url{http://gemini.readthedocs.org}.

%----------------------------------------------------------------------------------------

\newpage

%----------------------------------------------------------------------------------------
%	REFERENCES
%----------------------------------------------------------------------------------------

\section{References}

%TODO Change to using BiBTeX
\begin{enumerate}
  \item Paila U, Chapman BA, Kirchner R and Quinlan AR. "GEMINI: Integrative Exploration of Genetic Variation and Genome Annotations". PLoS Comput Biol, 2013, 9(7): e1003153. doi:10.1371/journal.pcbi.1003153
  \item McLaren W, Pritchard B, Rios D, Chen Y, Flicek P and Cunningham F. "Deriving the consequences of genomic variants with the Ensembl API and SNP Effect Predictor". Bioinformatics, 2010, 26(16):2069-70, doi:10.1093/bioinformatics/btq330
\end{enumerate}

\section{Acknowledgements}
This tutorial is an adaptation of the one created by Louis letourneau \url{https://github.com/lletourn/Workshops/tree/ebiCancerWorkshop201407doc/01-SNVCalling.md}. I would like to thank and acknowledge Louis for this help and for sharing his material. The format of the tutorial has been inspired from Mar Gonzalez Porta. I also want to acknowledge Joel Fillon, Louis Letrouneau (again), Francois Lefebvre, Maxime Caron and Guillaume Bourque for the help in building these pipelines and working with all the various datasets.
